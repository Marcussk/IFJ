\documentclass[a4paper]{article}
\usepackage[left=2cm,right=2cm,top=2.5cm,bottom=2.5cm,footnotesep=0.5cm]{geometry}
\usepackage[utf8x]{inputenc}
\usepackage[czech]{babel}
\usepackage{pdflscape}
\usepackage{graphicx}


\begin{document}
\begin{titlepage}
\begin{center}
\textsc{\Huge{Vysoké učení technické v~Brně}}\\
\medskip
\huge{Fakulta informačních technologií}\\
\vspace{\stretch{0.382}}
\Large{Dokumentácia  k projektu IFJ2014}\\
\medskip
\Huge{Tím 076, varianta b/2/II}\\
\vspace{\stretch{0.618}}
\end{center}

\noindent
Rozšírenia: \\
FUNEXP, REPEAT, MSG, MINUS
\newline
Tím:\\
Michal Orsák (xorsak01) : 20\%  (vedúci) \\
Milan Skála (xskala09) : 20\% \\
Marek Beňo (xbenom01) : 20\% \\
Zuzana Studená (xstude22) : 20\% \\
Roman Sichkaruk (xsichk00) : 20\% 
\end{titlepage}
\indent

\tableofcontents

\newpage
\section{Úvod}
Táto dokumentácia popisuje projekt IFJ2014 (varianta b/2/II) k predmetu Formální jazyky a prekladače.\\
V rámcií tejto varianty sme implementovali:
\begin{itemize}
\item{Hashovacia tabuľka}
\item{Algoritmus Heap sort}
\item{Algoritmus Boyer-Moore}
\end{itemize} 
Samotná implementácia je ďalej rozdelená na nasledovné časti:
\begin{itemize}
\item{Lexikálna analýza}
\item{Syntaktická a sémantická analýza}
\item{Interpretácia}
\end{itemize}
Všetky tieto časti budú ďalej spomenuté a rozšírené.

\section{Lexikálna analýza}
Lexikálny analyzátor je reprezentovaný konečným automatom, ktorý pomocou aktuálneho stavu a načítaného znaku vracia buď lexikálnu chybu alebo token. Úloha lexikálnej analýzy spočíva v:
\begin{itemize}
\item{Odstránenie komentárov}
\item{Rozdelenie vstupu na tokeny pre ďalšie spracovanie syntaktickou analýzou}
\end{itemize}
Lexikálny analyzátor pijíma ako vstup zdrojový program, ktorý spracováva. Počas spracovávania programu vždy syntaktická analýza žiada o ďalší token na spracovanie.

Pre potreby implementácie obsahuje lexikálny analyzér možnosť vrátiť spracovaný token a uložiť ho do bufferu. Spracovávanie ďalej nepokračuje načítavaním zo súboru ale z tohoto bufferu.

\section{Syntaktická a sémantická analýza}
Úlohou syntaktickej analýzi je skontrolovať konštrukcie vstupného súboru a detekovať nesúlady so zadaním.\\
Syntaktickú analýzu delíme na dve časti kvôli rozdielnemu spôsobu spracovania.

\subsection{Metóda rekurzívneho zostupu}
Rekurzívny zostup je jadrom syntaktickej analýzy. Zhora nadol spracováva tokeny z lexikálnej analýzy pomocou pravidiel LL gramatiky. \\
Implementácia spočíva z hlavnej funkcie, ktorá podľa prijatého tokenu vyberie typ konštrukcie na spracovanie. Nasleduje volanie funkcie v ktorej sa spracováva daný blok programu. Týmto spôsobom sa rekurzívne spracuje program.\\
V prípade potreby spracovávať výrazy nastáva prepnutie do spracovávania precedenčnou analýzou. Nastáva volanie funkcie a následné spracovávanie ďalších tokenov až po návrat z tejto funkcie.

\subsubsection{LL gramatika}
LL gramatika predstavuje súbor pravidiel podľa ktorých pracuje syntaktická analýza. LL predstavuje anglicky: left-to-right a leftmost, teda spracovávanie zľava doprava a zpracovávanie najľavejšieho neterminálu.\\
V tomto projekte je konkrétne využitá LL(1) gramatika, ktorá vyžaduje tieto vlastnosti:
\begin{enumerate}
\item{Neobsahuje nejednoznačnosť\\
Pri každom pravidle je teda nutné aby bolo na základe aktuálneho stavu možno spracovávať práve jedno pravidlo}
\item{Nemá ľavú rekurziu\\
Pri nahrádzaní najľavejšieho neterminálu nesmie dôjsť k nekonečnému cyklu a to priamo alebo nepriamo}
\item{Je lookahead(1)\\
Rozvoj pravidiel závisí od nasledujúceho tokenu a nemožno sa rozhodovať podľa viacerých tokenov z budúcnosti}
\end{enumerate}

\subsubsection{LL tabuľka}
Tabuľka založená na LL gramatike pre výber pravidla.\\
Stĺpce predstavujú nasledujúci token a riadky najľavejší neterminál. Prienikom aktuálneho stavu je identifikátor pravidla, ktoré sa použije.

\subsection{Metóda precedenčnej analýzy}
Spracováva výrazy syntaktickou analýzou zdola nahor. K spracovávaniu výrazov dochádza pri prepnutí sa z syntaktickej analýzy a končí sa pri načítaní tokenu nepatriacemu výrazu. Pre svoje fungovanie využíva výrazová analýza svoj vlastný zásobník a simuluje redukčné pravidlá:
\begin{itemize}
\item{E $\rightarrow$ E + E}
\item{E $\rightarrow$ E - E}
\item{E $\rightarrow$ E * E}
\item{E $\rightarrow$ E / E}
\item{E $\rightarrow$ E $<$ E}
\item{E $\rightarrow$ E $>$ E}
\item{E $\rightarrow$ E $<=$ E}
\item{E $\rightarrow$ E $>=$ E}
\item{E $\rightarrow$ E = E}
\item{E $\rightarrow$ ( E )}
\item{E $\rightarrow$ i}
\item{E $\rightarrow$ f()}
\item{E $\rightarrow$ f(e)}
\item{E $\rightarrow$ f(E, E, ..., E)}
\end{itemize}
Pravidlá nie sú pevne implementované ale prebieha ich simulácia pomocou výberu redukčného pravidla.
Výber redukčného pravidla prebieha podľa priority (precedencie) symbolov.

\subsubsection{Precedenčná tabuľka}
Precedenčná tabuľka slúži na výber operácie shift alebo reduce. Je vytvorená na základe precedencie operátorov daného jazyka. Pozostáva zo všetkých možných operátorov nad danými výrazmi a príznakmi pre operáciu nad zásobníkom: shift,reduce,equal a chybový stav.

\subsection{Sémantická analýza}
Úlohou Semantickej analýzy je kontrola dátových typov pri operáciách s premennými a ich prípadné pretypovanie.

Funkcia checktypes je využívaná na kontrolu dátových typov pri priraďovaní.\\
Funkcia checkcond sa používa na kontrolu správneho typu podmienky.\\
Funkcia typeconvert sa používa na pretypovanie v prípade operácie delenia.\\
Kontrola inicializácie premmenej je spracovaná kontrolou príznaku inicializácie v štruktúre premennej.


\section{Interpretácia}
Samotný Interpret je reprezentovaný štruktúrou, ktorá pozostáva z fronty pre inštrukcie a zásobníku pre vlastné spracovanie. Interpret prijíma trojadresný kód vo forme inštrukcií zoradených vo fronte a pomocou vlastného zásobníku ich spracováva. V prípade, že nenastane behová chyba výsledkom je program v~jazyku C.

Interpret spracováva typ inštrukcie a jej kód podľa výčtových typov a jednotlivé operandy funkcie a cieľ ako odkazy na parametre. Pre prípad zanorenia sa do funkcií je riešená relatívna pozícia pomocou offsetu na zásobníku.

\section{Algoritmy}
Zadanie projektu vyžaduje implementáciu tabulky symbolov rovnako ako aj algoritmov pre vyhľadávanie a radenie. Implementácie týchto častí sú popísané nižšie. 

\subsection{Hashovacia tabuľka}
Hashovacia tabuľka je abstraktný dátový typ, ktorý využívame na implementáciu tabuľky symbolov. \\
Hashovacia tabuľka sa vyznačuje náročnoťou na pamäť $\mathcal{O}(n)$ avšak vykazuje priemernú náročnosť pre vkladanie a vyhľadávanie záznamov $\mathcal{O}(1)$. Pri zaplnení tabuľky symbolov je potreba túto tabuľku zväčšiť a následne realokovať. Implementácia tabuľky využíva velkosť mocniny $2$ kvôli zníženiu nároku alokovania v prípade nedostatku pamäti.\\
Tabuľka slúži na ukladanie a následné vyhľadávanie symbolov pomocou hashovacej funkcie. Funkcia slúži na priradenie unikátnej hodnoty pre akýkoľvek daný klúč. Pomocou hodnoty vrátenej hashovaciou funkciou sa vyhľadáva alebo vkladá na danú pozíciu tabuľky.\\ 
Tabuľka rozširuje lokálne a globálne symboly pomocou zanorovaňia sa do novej úrovne tejto tabuľky. Zanorovanie a vynorovanie prebieha pri definícií/deklarácií funkcie alebo návratu z nej. 

\subsection{Heap sort}
Využili sme pole pre implementáciu vyhľadávania v halde. Napriek nutnosti spravovať haldu a jeho nestabilnosti sa algoritmus vyznačuje malými nárokmi. Vyznačuje sa najhoršou maximálnou náročnosťou $\mathcal{O}N\log{N}$.
Pri tejto implementácií považujeme prvok str[0] za vrchol haldy. Potom platí: str[2i+1] pre ľavého potomka a str[2i+2] pre pravého potomka. 
\subsubsection{Implementácia}
V prvom priebehu využitie funkcie Sift na zostavenie hromady. Vytvára sa teda $N / 2$ uzlov a následne sa do všetkých N prvkov vloží hodnota. Do koreňového uzlu sa vloží najväčšia hodnota. \\
V nasledujúcich priebehoch sa využíva striedavé volanie funkcií Sift a Change na zoradenie prvku s najvyššou hodnotou a následné vytvorenie hromady. 
\subsection{Boyer-Moore} 
V tomto projekte bol implementovaný vyhľadávací algoritmus Boyer-Moore. Medzi jeho hlavné črty patrí:
\begin{itemize}
\item{Vyhľadávanie zprava doľava}
\item{Počet hľadaní v najlepšom prípade: $\mathcal{O}(n / m)$}
\item{Počet hľadaní v najhoršom prípade: 3n pri neperiodickom vzore}
\end{itemize}

Hlavná funkcia používa dve heurestiky na predpočítanie. Využíva  sa čas použitý na predpočítanie k zrýchleniu behu samotnéhu vyhľadávania. Používanie vzoru pri predpočítavaní znamená nákladnosť programu. Funkcia vyberá vyššiu hodnotu z obidvoch heurestík pre minimalizáciu počtu hľadaní.

\subsubsection{Heurestiky}
Bad Character\\
Predpočítava pole znakov pre každý znak abecedy. Toto pole sa naplní dĺžkou reťazca, ktorá sa využíva v~prípade, že nastane nezhoda znaku a textu. V prípade že nájdený znak patrí hľadanému reťazcu uloží sa do pola jeho vzdialenosť od konca vzorca. Na konci predpočítavania je pole naplnené hodnotami s~možnou dĺžkou skoku v prípade že sa reťazec a text po prvom hľadaní nezhodujú.

Good Suffix\\
Používa dve polia pre prípady kedy nastane a nenastane zhoda v hľadanom vzorci a texte. Heurestika využíva znalosť všetkých možných suffixov daného vzoru a ukladá si hranice daného substringu. V prípade že vznikne rozdiel medzi subreťazcami posunie sa na ďalší výskyt subreťazca, ktorému prechádzca znak, ktorý sa nezhoduje. Pokiaľ taký segment neexistuje, zarovná sa najdlhší zhodujúci sa suffix reťazca a prefix vzoru.

\section{Práca v tíme}
Keďže projekt sa vypracováva ako skupinový je potreba komunikácie a spolupráce v tíme. Rýchla komunikácia a možnosť spolupráce aj bez vlastného osobného stretnutia boli pre prácu na tomto projekte klúčové.

\subsection{Verzovací systém}
Pred samotným zahájením práce na projekte bol ustanovený verzovací systém git a vytvorený repozitár na stránkach github. Nasledujúce zoznámenie tímu s princípami verzovacieho systému bolo základom pre ďalšiu prácu na projekte. Výhodou tohoto riešenia bola nielen možnosť pracovať na rovnakých súboroch a s jednoduchosťou riešiť kolízie ale aj vlastné zálohovanie už dokončenej práce.
 
\subsection{Rozdelenie práce}
Pre potreby projektu sme využili osobné stretnutia vďaka ktorým sme v tíme dostali celkovú predstavu ako si prácu predstavujeme, ustanovili zásady práce do budúcna a začali získavať vedomosti potrebné pre samotné programovanie. Neskôr boli využité hlavne skupinové konverzácie a skupinové hovory cez internet. Toto sa ukázalo ako efektívna voľba, ktorá šetrila čas cestovaním a umožňovala rýchlu deľbu práce a možnosť opravovať a riešiť aktuálne problémy.

Rozdelenie práce na 4 samostatné rovnaké celky sa nám nepodarilo aplikovať a využívali sme teda dynamické prideľovanie úloh a rozdelenie práce podľa akútnych problémov. Neraz nastávala situácia keď na jednom probléme pracovalo viac ludí spoločne alebo dokonca celý tím a preto je problematické ustanoviť presný zoznam rozdelenia práce.

\clearpage
\begin{landscape}
\section{Model konečného automatu}
%\resizebox{\textwidth}{!}{\includegraphics{lexparser}}
%0.42 0.35
\scalebox{0.40}[0.38]{\includegraphics{lexparser}}
\end{landscape}
\clearpage

\newpage
\section{LL gramatika}
%\noindent
\begin{enumerate}
\item{START $\rightarrow $ var VARINIT FUNCDECLARE PROGRAM}
\item{START $\rightarrow $ FUNCDECLARE PROGRAM}
\item{VARINIT $\rightarrow $ VARDECLARE VARDECLARES } 
\item{VARDECLARES $\rightarrow $ id : TYP ; } 
\item{VARDECLARES $\rightarrow \epsilon$ } 
\item{VARDECLARES $\rightarrow$ VARDECLARE VARDECLARES } 
\item{PROGRAM $\rightarrow$ begin S end. } 
\item{S $\rightarrow$ IF } 
\item{S $\rightarrow$ WHILE } 
\item{S $\rightarrow$ id := E } 
\item{S $\rightarrow$ STMTLIST } 
\item{VYRAZ $\rightarrow$ id := E } 
\item{VYRAZ $\rightarrow$ E } 
\item{STMTLIST $\rightarrow$ begin INIT end } 
\item{INIT $\rightarrow$ VYRAZ POSTVYRAZ } 
\item{INIT $\rightarrow$ BLOCK INIT } 
\item{INIT $\rightarrow \epsilon$ } 
\item{BLOCK $\rightarrow$ IF } 
\item{BLOCK $\rightarrow$ WHILE }  
\item{POSTVYRAZ $\rightarrow$ ; POSTSTREDNIK } 
\item{POSTVYRAZ $\rightarrow \epsilon$  } 
\item{POSTSTREDNIK $\rightarrow$ BLOCK INIT } 
\item{POSTSTREDNIK $\rightarrow$ VYRAZ POSTVYRAZ } 
\item{IF $\rightarrow$ if E then STMTLIST else STMTLIST } 
\item{WHILE $\rightarrow$ while EXPR do STMTLIST } 
\item{FUNCDECLARE $\rightarrow \epsilon$ } 
\item{FUNCDECLARE $\rightarrow$ FUNC FUNCDECLARE } 
\item{FUNC $\rightarrow$ function fcid ( PARAMS ) : T; FORWARD VARINIT FLIST } 
\item{FORWARD $\rightarrow$ forward ; } 
\item{FORWARD $\rightarrow \epsilon$ } 
\item{PARAMS $\rightarrow \epsilon$ } 
\item{PARAMS $\rightarrow$ VARINIT } 
\item{FLIST $\rightarrow$ begin INIT end; }
\end{enumerate}

%\newpage
\begin{landscape}
\section{LL tabuľka}
\begin{table}[h]
\centering
\begin{tabular}{|l|l|l|l|l|l|l|l|l|l|l|l|l|l|l|l|l|l|l|l|l|l|l|}
\hline
             & var & id & : & T & ; & begin & end. & := & E & end & if & then & else & while & do & function & fcid & ( & )  & forward & end; & \$ \\ \hline
START        & 1   &    &   &     &   & 2     &      &    &      &     &    &      &      &       &    & 2        &      &   &    &         &      &  \\ \hline
VARINIT      &     & 3  &   &     &   &       &      &    &      &     &    &      &      &       &    &          &      &   &    &         &      &  \\ \hline
VARDECLARE   &     & 4  &   &     &   &       &      &    &      &     &    &      &      &       &    &          &      &   &    &         &      &  \\ \hline
VARDECLARES  &     & 6  &   &     &   & 5     &      &    &      &     &    &      &      &       &    & 5        &      &   & 5  &         &      &  \\ \hline
PROGRAM      &     &    &   &     &   & 7     &      &    &      &     &    &      &      &       &    &          &      &   &    &         &      &  \\ \hline
S            &     & 10 &   &     &   & 11    &      &    &      &     & 8  &      &      & 9     &    &          &      &   &    &         &      &  \\ \hline
VYRAZ        &     & 12 &   &     &   &       &      &    & 13   &     &    &      &      &       &    &          &      &   &    &         &      &  \\ \hline
STMTLIST     &     &    &   &     &   & 14    &      &    &      &     &    &      &      &       &    &          &      &   &    &         &      &  \\ \hline
INIT         &     & 15 &   &     &   &       &      &    & 15   & 17  & 16 &      &      & 16    &    &          &      &   &    &         & 17   &  \\ \hline
BLOCK        &     &    &   &     &   &       &      &    &      &     & 18 &      &      & 19    &    &          &      &   &    &         &      &  \\ \hline
POSTVYRAZ    &     & 23 &   &     &   &       &      &    &      & 21  &    &      &      &       &    &          &      &   &    &         & 21   &  \\ \hline
POSTSTREDNIK &     &    &   &     &   &       &      &    & 23   &     & 22 &      &      & 22    &    &          &      &   &    &         &      &  \\ \hline
IF           &     &    &   &     &   &       &      &    &      &     & 24 &      &      &       &    &          &      &   &    &         &      &  \\ \hline
WHILE        &     &    &   &     &   &       &      &    &      &     &    &      &      & 25    &    &          &      &   &    &         &      &  \\ \hline
FUNCDECLARE  &     &    &   &     &   & 26    &      &    &      &     &    &      &      &       &    & 27       &      &   &    &         &      &  \\ \hline
FUNC         &     &    &   &     &   &       &      &    &      &     &    &      &      &       &    & 28       &      &   &    &         &      &  \\ \hline
FORWARD      &     & 30 &   &     &   &       &      &    &      &     &    &      &      &       &    &          &      &   &    & 29      &      &  \\ \hline
PARAMS       &     & 32 &   &     &   &       &      &    &      &     &    &      &      &       &    &          &      &   & 31 &         &      &  \\ \hline
FLIST        &     &    &   &     &   & 33    &      &    &      &     &    &      &      &       &    &          &      &   &    &         &      &  \\ \hline
\end{tabular}
\end{table}
\end{landscape}

\section{Precedenčná tabuľka}
\begin{table}[h]
\centering
\begin{tabular}{|l|l|l|l|l|l|l|l|l|l|l|l|l|l|l|l|l|}
\hline
                      & + & - & * & / & \textless & \textgreater & \textless= & \textgreater= & = & \textless\textgreater & ( & ) & , & \$ & i & f \\ \hline
+                     & R & R & S & S & R         & R            & R          & R             & R & R                     & S & R & R & R & S & S \\ \hline
-                     & R & R & S & S & R         & R            & R          & R             & R & R                     & S & R & R & R & S & S \\ \hline
*                     & R & R & R & R & R         & R            & R          & R             & R & R                     & S & R & R & R & S & S \\ \hline
/                     & R & R & R & R & R         & R            & R          & R             & R & R                     & S & R & R & R & S & S \\ \hline
\textless             & S & S & S & S & R         & R            & R          & R             & R & R                     & S & R & R & R & S & S \\ \hline
\textgreater          & S & S & S & S & R         & R            & R          & R             & R & R                     & S & R & R & R & S & S \\ \hline
\textless=            & S & S & S & S & R         & R            & R          & R             & R & R                     & S & R & R & R & S & S \\ \hline
\textgreater=         & S & S & S & S & R         & R            & R          & R             & R & R                     & S & R & R & R & S & S \\ \hline
=                     & S & S & S & S & R         & R            & R          & R             & R & R                     & S & R & R & R & S & S \\ \hline
\textless\textgreater & S & S & S & S &           & R            & R          & R             & R & R                     & S & R & R & R & S & S \\ \hline
(                     & S & S & S & S & S         & S            & S          & S             & S & S                     & E & R & R & R & X & R \\ \hline
)                     & R & R & R & R & R         & R            & R          & R             & R & R                     & E & R & R & R & X & R \\ \hline
,                     & S & S & S & S & S         & S            & S          & S             & S & S                     & S & E & E & X & S & S \\ \hline
\$                     & S & S & S & S & S         & S            & S          & S             & S & S                     & S & X & X & X & S & S \\ \hline
i                     & R & R & R & R & R         & R            & R          & R             & R & R                     & X & R & R & R & X & X \\ \hline
f                     & X & X & X & X & X         & X            & X          & X             & X & X                     & E & X & X & X & X & X \\ \hline
\end{tabular}
\end{table}
\end{document}

